\documentclass[../../main.tex]{subfiles}

\begin{document}
\section{Introduzione}

\setLayout{mainpoint}  
\begin{frame}
    \frametitle{Introduzione}
\end{frame}

\subsection{Contesto di riferimento}
\setLayout{horizontal} 
\begin{frame}
    \frametitle{Contesto di riferimento}
    
    \begin{columns}
        
        \column{0.6\textwidth}
        \begin{itemize}
            \item L'11 marzo 2020, l'Organizzazione Mondiale della Sanità (OMS) ha dichiarato che il focolaio internazionale di infezione da nuovo coronavirus SARS-CoV-2 è una pandemia.
        \end{itemize}

        \column{0.4\textwidth}
        \begin{figure}
            \includegraphics[width= .7\columnwidth]{virus}
            \caption{Illustrazione di SARS-CoV-2 ad opera del Centers for Disease Control and Prevention (\href{https://phil.cdc.gov/Details.aspx?pid=23312}{Fonte}).}
        \end{figure}
        
    \end{columns}

\end{frame}

\begin{frame}
    \frametitle{Misure di contrasto}

    \begin{itemize}
        \item<1-> Sicuramente la medicina e la ricerca hanno e stanno svolgendo
        un ruolo primario;
        \item<2-> tuttavia molto importanti sono anche i media e i vari canali informativi;
        \begin{itemize}
            \item sono responsabili della comprensione e \alert{consapevolezzazione dei cittadini};
            \item uno degli strumenti tecnologici di maggior autorevolezza e adozione è la dashboard ufficiale del Dipartimento della Protezione Civile (\href{https://opendatadpc.maps.arcgis.com/apps/opsdashboard/index.html\#/b0c68bce2cce478eaac82fe38d4138b1}{COVID-19 Situazione Italia}).
        \end{itemize}              

    \end{itemize}

\end{frame}

\subsection{Problematiche individuate}

\begin{frame}
    \frametitle{Problematiche individuate}
    Abbiamo analizzato in dettaglio la sua interfaccia e abbiamo rinvenuto diverse criticità.\\
    Nelle slide successive esse saranno dettagliate, ora possono essere riassunte in:
    \begin{itemize}
        \item<1-> assenza di contestualizzazione dei dati mostrati;
        \item<2-> assenza di molte informazioni essenziali;
        \item<3-> scarsa usabilità ed esperienza utente insoddisfacente;
        \item<4-> inconsistenze nell'interazione e assenza di meccanismi di prevenzione degli errori.
    \end{itemize}

\end{frame}

\begin{frame}
    \frametitle{Problematiche individuate}
    La dashboard del DPC è una delle fonti di riferimento principali dei giornalisti che scrivono articoli sull'andamento del Covid-19.
    \vspace{-50pt}
    \begin{alertblock}{Comprensione superficiale dei giornalisti}
        Le criticità della sua interfaccia non permettono ai giornalisti di maturare una conoscenza completa e profonda del quadro epidemiologico: inoltre, molti aggregati essenziali non sono comunicati, per cui sono costretti a calcolarli manualmente ogni volta.
    \end{alertblock}

\end{frame}

\begin{frame}
    \frametitle{Problematiche individuate}
    La dashboard del DPC è la fonte di riferimento dei giornalisti che scrivono articoli sull'andamento del Covid-19.
    \vspace{-50pt}
    \begin{alertblock}{Articoli asettici}
        I giornalisti, quindi, offrono ai loro lettori articoli asettici, dal povero contenuto informativo: la maggior parte è un elenco dei valori assoluti delle metriche epidemiologiche, completamente de-contestualizzati e, alle volte, errati.
    \end{alertblock}

\end{frame}


\subsection{Soluzione proposta}

\begin{frame}
    \frametitle{Soluzione proposta}
    Abbiamo riprogettato l'interfaccia della dashboard del DPC.

    \vspace{-40pt}
    \begin{block}{Focus sui giornalisti}
        Abbiamo focalizzato la riprogettazione sul segmento di utenza dei giornalisti che comunicano i dati epidemiologici e che non hanno necessariamente un background scientifico: sono un anello essenziale per la consapevolizzazione dei cittadini, per cui crediamo sia fondamentale realizzare un'interfaccia pensata specificatamente per le loro esigenze.
    \end{block}

\end{frame}

\begin{frame}
    \frametitle{Soluzione proposta}
    
    La parola chiave è \alert{contestualizzazione}:
    \begin{itemize}
        \item ogni valore viene presentato nel contesto spaziale e temporale in cui si manifesta: ai valori assoluti sono sempre affiancati rapporti relativi, rapporti percentuali, grafici temporali ecc;
        \item è l'unico modo per comprenderlo a fondo e comunicarlo in maniera semplice e puntuale.
    \end{itemize}
    

\end{frame}

\end{document}